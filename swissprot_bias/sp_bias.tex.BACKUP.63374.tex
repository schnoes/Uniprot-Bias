<<<<<<< HEAD
%% BioMed_Central_Tex_Template_v1.05
%%                                      %
%  bmc_article.tex            ver: 1.05 %
%                                       %


%%%%%%%%%%%%%%%%%%%%%%%%%%%%%%%%%%%%%%%%%
%%                                     %%
%%  LaTeX template for BioMed Central  %%
%%     journal article submissions     %%
%%                                     %%
%%         <27 January 2006>           %%
%%                                     %%
%%                                     %%
%% Uses:                               %%
%% cite.sty, url.sty, bmc_article.cls  %%
%% ifthen.sty. multicol.sty		       %%
%%									   %%
%%                                     %%
%%%%%%%%%%%%%%%%%%%%%%%%%%%%%%%%%%%%%%%%%


%%%%%%%%%%%%%%%%%%%%%%%%%%%%%%%%%%%%%%%%%%%%%%%%%%%%%%%%%%%%%%%%%%%%%
%%                                                                 %%	
%% For instructions on how to fill out this Tex template           %%
%% document please refer to Readme.pdf and the instructions for    %%
%% authors page on the biomed central website                      %%
%% http://www.biomedcentral.com/info/authors/                      %%
%%                                                                 %%
%% Please do not use \input{...} to include other tex files.       %%
%% Submit your LaTeX manuscript as one .tex document.              %%
%%                                                                 %%
%% All additional figures and files should be attached             %%
%% separately and not embedded in the \TeX\ document itself.       %%
%%                                                                 %%
%% BioMed Central currently use the MikTex distribution of         %%
%% TeX for Windows) of TeX and LaTeX.  This is available from      %%
%% http://www.miktex.org                                           %%
%%                                                                 %%
%%%%%%%%%%%%%%%%%%%%%%%%%%%%%%%%%%%%%%%%%%%%%%%%%%%%%%%%%%%%%%%%%%%%%


\NeedsTeXFormat{LaTeX2e}[1995/12/01]
\documentclass[10pt]{bmc_article}    



% Load packages
\usepackage{cite} % Make references as [1-4], not [1,2,3,4]
\usepackage{url}  % Formatting web addresses  
\usepackage{ifthen}  % Conditional 
\usepackage{multicol}   %Columns
\usepackage[utf8]{inputenc} %unicode support
%\usepackage[applemac]{inputenc} %applemac support if unicode package fails
%\usepackage[latin1]{inputenc} %UNIX support if unicode package fails
\urlstyle{rm}
 
 
%%%%%%%%%%%%%%%%%%%%%%%%%%%%%%%%%%%%%%%%%%%%%%%%%	
%%                                             %%
%%  If you wish to display your graphics for   %%
%%  your own use using includegraphic or       %%
%%  includegraphics, then comment out the      %%
%%  following two lines of code.               %%   
%%  NB: These line *must* be included when     %%
%%  submitting to BMC.                         %% 
%%  All figure files must be submitted as      %%
%%  separate graphics through the BMC          %%
%%  submission process, not included in the    %% 
%%  submitted article.                         %% 
%%                                             %%
%%%%%%%%%%%%%%%%%%%%%%%%%%%%%%%%%%%%%%%%%%%%%%%%%                     


\def\includegraphic{}
\def\includegraphics{}



\setlength{\topmargin}{0.0cm}
\setlength{\textheight}{21.5cm}
\setlength{\oddsidemargin}{0cm} 
\setlength{\textwidth}{16.5cm}
\setlength{\columnsep}{0.6cm}

\newboolean{publ}

%%%%%%%%%%%%%%%%%%%%%%%%%%%%%%%%%%%%%%%%%%%%%%%%%%
%%                                              %%
%% You may change the following style settings  %%
%% Should you wish to format your article       %%
%% in a publication style for printing out and  %%
%% sharing with colleagues, but ensure that     %%
%% before submitting to BMC that the style is   %%
%% returned to the Review style setting.        %%
%%                                              %%
%%%%%%%%%%%%%%%%%%%%%%%%%%%%%%%%%%%%%%%%%%%%%%%%%%
 

%Review style settings
\newenvironment{bmcformat}{\begin{raggedright}\baselineskip20pt\sloppy\setboolean{publ}{false}}{\end{raggedright}\baselineskip20pt\sloppy}

%Publication style settings
%\newenvironment{bmcformat}{\fussy\setboolean{publ}{true}}{\fussy}



% Begin ...
\begin{document}
\begin{bmcformat}


%%%%%%%%%%%%%%%%%%%%%%%%%%%%%%%%%%%%%%%%%%%%%%
%%                                          %%
%% Enter the title of your article here     %%
%%                                          %%
%%%%%%%%%%%%%%%%%%%%%%%%%%%%%%%%%%%%%%%%%%%%%%

\title{Source of Functional Annotations in UniProtKB and implications for
Function Prediction}
 
%%%%%%%%%%%%%%%%%%%%%%%%%%%%%%%%%%%%%%%%%%%%%%
%%                                          %%
%% Enter the authors here                   %%
%%                                          %%
%% Ensure \and is entered between all but   %%
%% the last two authors. This will be       %%
%% replaced by a comma in the final article %%
%%                                          %%
%% Ensure there are no trailing spaces at   %% 
%% the ends of the lines                    %%     	
%%                                          %%
%%%%%%%%%%%%%%%%%%%%%%%%%%%%%%%%%%%%%%%%%%%%%%


\author{Alexandra Schnoes$^1$%
       \email{Alexandra Schnoes - schnoes@ucsf.edu}%
      \and
         Alexander Thorman$^2$%
         \email{Alexander Thorman - thormanaaw@muohio.edu}
       \and 
         Patricia C Babbitt$^1$%
         \email{Patricia C Babbitt - babbitt@ucsf.edu}%
        and
         Iddo Friedberg\correspondingauthor$^{2,3}$%
        \email{Iddo Friedberg\correspondingauthor - i.friedberg@muohio.edu}
      }
      

%%%%%%%%%%%%%%%%%%%%%%%%%%%%%%%%%%%%%%%%%%%%%%
%%                                          %%
%% Enter the authors' addresses here        %%
%%                                          %%
%%%%%%%%%%%%%%%%%%%%%%%%%%%%%%%%%%%%%%%%%%%%%%

\address{%
    \iid(1)UCSF,%
        San Francisco, CA, USA\\
    \iid(2)Department of Microbiology, Miami University, Oxford, OH USA\\
    \iid(3)Department of Computer Science and Software Engineering , Miami University, Oxford, OH USA
}%

\maketitle

%%%%%%%%%%%%%%%%%%%%%%%%%%%%%%%%%%%%%%%%%%%%%%
%%                                          %%
%% The Abstract begins here                 %%
%%                                          %%
%% The Section headings here are those for  %%
%% a Research article submitted to a        %%
%% BMC-Series journal.                      %%  
%%                                          %%
%% If your article is not of this type,     %%
%% then refer to the Instructions for       %%
%% authors on http://www.biomedcentral.com  %%
%% and change the section headings          %%
%% accordingly.                             %%   
%%                                          %%
%%%%%%%%%%%%%%%%%%%%%%%%%%%%%%%%%%%%%%%%%%%%%%


\begin{abstract}
        % Do not use inserted blank lines (ie \\) until main body of text.
        \paragraph*{Background:} Computational protein function
prediction programs rely upon well-annotated databases for training
their algorithms. These databases, in turn, rely upon the work of curators
applying experimental findings from scientific literature to protein
sequence data. However, due to various high-throughput experimental
assays, there are a few papers that dominate the protein annotations.
Here we investigate just how prevalent is the ``few papers --
many proteins'' bias. We discuss how this bias affects our view of the
protein function universe, and consequently our ability to predict
protein function.
      
        \paragraph*{Results:} We examine the annotation of UniProtKB by the Gene
Ontology Annotation project (GOA), and show that the distribution of proteins
annotated per paper follows a scale-free distribution, with X papers dominating X\%
of the annotations. Since each of the dominant papers describes the use of an assay
that can find only one function or a small group of functions, this leads to a
substantial bias in what we know about the function of many proteins.

        \paragraph*{Conclusions:} Given the experimental techniques
available, the protein function annotation bias is unavoidable. Knowing
that this bias exists and understanding its extent is important for
database curators, developers of function annotation programs, and
anyone who uses protein function annotation data to plan experiments.

\end{abstract}



\ifthenelse{\boolean{publ}}{\begin{multicols}{2}}{}




%%%%%%%%%%%%%%%%%%%%%%%%%%%%%%%%%%%%%%%%%%%%%%
%%                                          %%
%% The Main Body begins here                %%
%%                                          %%
%% The Section headings here are those for  %%
%% a Research article submitted to a        %%
%% BMC-Series journal.                      %%  
%%                                          %%
%% If your article is not of this type,     %%
%% then refer to the instructions for       %%
%% authors on:                              %%
%% http://www.biomedcentral.com/info/authors%%
%% and change the section headings          %%
%% accordingly.                             %% 
%%                                          %%
%% See the Results and Discussion section   %%
%% for details on how to create sub-sections%%
%%                                          %%
%% use \cite{...} to cite references        %%
%%  \cite{koon} and                         %%
%%  \cite{oreg,khar,zvai,xjon,schn,pond}    %%
%%  \nocite{smith,marg,hunn,advi,koha,mouse}%%
%%                                          %%
%%%%%%%%%%%%%%%%%%%%%%%%%%%%%%%%%%%%%%%%%%%%%%




%%%%%%%%%%%%%%%%
%% Background %%
%%
\section*{Background}
 Text for this section.

Functional annotation of proteins is a cardinal challenge in molecular biology today. It has
been said that while we may soon reach the \$1,000 genome (and for some prokaryotes, indeed
we have), we will still need to pay \$20,000 analysis bill. There are and estimated XXX
non-redundant open reading frame sequences, but less than 40\% are functionally annotated.
Moreover, the majority of functional annotations 


 
%%%%%%%%%%%%%%%%%%%%%%%%%%%%
%% Results and Discussion %%
%%
\section*{Results and Discussion}
  \subsection*{Papers and proteins}

    As described in the Background section, with the advent of
high-throughput experiments it has become possible to conduct
large-scale interrogations of protein functions. Some papers therefore
reveal one or more functional aspects of a large amount of proteins
which respond to the particular type of interrogation conducted. To
understand how prevalent this phenomenon is, we looked at the UniprotKB
gene ontology annotation files, or UniProt GOA. UniProtKB is annotated
by GO terms both manually and automatically using an exacting procedure
described in \cite{UniprotKB-GOA}. Briefly, there is a six-step
procedure which includes sequence curation, sequence motif analyses,
literature-based curation, reciprocal BLAST\cite{BLAST} searches,
attribution of all resources leading to the included findings, and a
quality assurance phase. If the annotation source is a research article, the
attribution includes a PubMed ID.  For each GO term associated with a
protein, there is also an \textit{evidence code} with which is used to
explain how the association between the protein and the GO term was
made. Experimental evidence codes include such terms as: Inferred by
Direct Assay (IDA) which indicates that ``a direct assay was carried out
to determine the function, process, or component indicated by the GO
term'' or \textit{Inferred from Physical Interaction} (IPI) which ``Covers
physical interactions between the gene product of interest and another
molecule.'' (Quotes are from the GO site, geneontology.org). The
computational analysis evidence codes are generally considered less
reliable than the experimental ones, and include terms such as
\textit{Inferred from Sequence or Structural Similarity} (ISS) and
\textit{Inferred from Sequence Orthology} (ISO). However, these are still
assigned by a curator. There are also
non-computational and non-experimental evidence codes, the most
prevalent being \textit{Inferred from Electronic Annotation} (IEA) ``Used
for annotations that depend directly on computation or automated
transfer of annotations from a database''. IEA evidence means that the
annotation was not made by a curator, and is not checked manually. 

Different degrees of reliability are associated with the  evidence codes, with
experimental codes considered to be of higher reliability than non-experimental
codes. However, due to an ongoing increase in high-throughput experimental
methods, we suspected that high-throughput experiments may dominate the protein
annotation landscape. 

To test our hypothesis, we first examined assignments with two degrees of reliability:
experimental evidence codes (EXP, IDA, IPI, IMP, IGI, IEP), and all others. The
reason for this partition into two groups is that we wanted to know which, if
any, high throughput experiments dominate the protein function annotation
landscape. This we compared with a baseline of annotations of all types of
evidence, and with those which are non-experimental. The results are shown in
Figure\ref{fig:papers-prots}.

As can be seen in Figure\ref{fig:papers-prots}, the distribution of proteins
annotated per paper follows a scale-free distribution.  Many proteins are annotated by a
few papers. (ZZZNeed to look at the percentage of proteins annotated by the top
10, top 20 papers.) We therefore conclude that there is indeed a bias in
experimental annotations, in which there are few papers that annotate a large
number of proteins.

Our next question was how much annotation bias these papers introduce. We decided
to look who are the top contributing papers to protein annotation in
UniProtKB-GOA, and what type of GO terms they contribute.  The results to the
first question are summarized in Table\ref{table:top-papers}.  As can be seen,
each paper is specific to a single species (typically a model
organism) and assay that is used to annotate the proteins in that organism. Since
a single assay was used, then typically only one ontology (MF, BP or CC) was
annotated. 
 
\subsection*{Annotation bias in model organisms}

To see how much a single species-- and method-- specific large-scale assay affects
the entire annotation of a species, we examined the relative contribution of each
paper to the entire corpus of experimentally annotated protein in that species.  All
the species we examined were model organisms, as all the top annotation-contributing
papers dealt with model organisms.  The results are summarized in
Figure\ref{fig:rel-contrib}.

\subsection*{Annotation quality}

One reflection on annotation coverage is the number of GO terms assigned to any
given protein. Ostensibly, the larger the number of GO terms that are assigned to a
protein, the more comprehensive its annotation, provided that these terms are
non-redundant, i.e. not direct parents of each other. 

The median number of annotations per protein was 1.09, which means that most of the top-50
papers do not provide more than a single GO-term per annotation. Also, the mean number of
annotations per protein was 1.59. The difference between the mean and median reflects that
most papers have an annotations-per-protein ratio which is close to 1, whereas few have a
higher annotation ratio. As shown in Figure~\ref{fig:annotation-ratio}, that is essentially
the case. Seven papers had an annotations per proteins ratio which is above 2. Those
were paper numbers 1, 4, 10, 23, 34 and 43 on the list. We decided to examine these papers
to see whether these studies did indeed provide better coverage, and what distinguished them
from the other high throughput studies. To follow are brief summaries of the
methodologies used in these papers.

\textbf{Toward a confocal subcellular atlas of the human proteome.}

In this microscopy-based study, the authors used specific staining techniques and
confocal microscopy to describe the subcellular localization of 4937 proteins, which
were each assigned to one or more of ten different subcellular compartments. Since
proteins may be assigned to more than a single compartment, there is a mean of 2.23
GO annotations per protein. 
\cite{PMID:11121744}. 


It appears that most of the high-annotating papers are 



%%%%%%%%%%%%%%%%%%%%%%
\section*{Conclusions}
  Text for this section \ldots


  
%%%%%%%%%%%%%%%%%%
\section*{Methods}
  \subsection*{Databases used}
    Text for this sub-section \ldots

  \subsection*{Another methods sub-heading for this section}
    Text for this sub-section \ldots

  \subsection*{Yet another sub-heading for this section}
    Text for this sub-section \ldots


    
%%%%%%%%%%%%%%%%%%%%%%%%%%%%%%%%
\section*{Authors contributions}
    Text for this section \ldots

    

%%%%%%%%%%%%%%%%%%%%%%%%%%%
\section*{Acknowledgements}
  \ifthenelse{\boolean{publ}}{\small}{}
  Text for this section \ldots


 
%%%%%%%%%%%%%%%%%%%%%%%%%%%%%%%%%%%%%%%%%%%%%%%%%%%%%%%%%%%%%
%%                  The Bibliography                       %%
%%                                                         %%              
%%  Bmc_article.bst  will be used to                       %%
%%  create a .BBL file for submission, which includes      %%
%%  XML structured for BMC.                                %%
%%                                                         %%
%%                                                         %%
%%  Note that the displayed Bibliography will not          %% 
%%  necessarily be rendered by Latex exactly as specified  %%
%%  in the online Instructions for Authors.                %% 
%%                                                         %%
%%%%%%%%%%%%%%%%%%%%%%%%%%%%%%%%%%%%%%%%%%%%%%%%%%%%%%%%%%%%%


{\ifthenelse{\boolean{publ}}{\footnotesize}{\small}
 \bibliographystyle{bmc_article}  % Style BST file
  \bibliography{bmc_article} }     % Bibliography file (usually '*.bib' ) 

%%%%%%%%%%%

\ifthenelse{\boolean{publ}}{\end{multicols}}{}

%%%%%%%%%%%%%%%%%%%%%%%%%%%%%%%%%%%
%%                               %%
%% Figures                       %%
%%                               %%
%% NB: this is for captions and  %%
%% Titles. All graphics must be  %%
%% submitted separately and NOT  %%
%% included in the Tex document  %%
%%                               %%
%%%%%%%%%%%%%%%%%%%%%%%%%%%%%%%%%%%

%%
%% Do not use \listoffigures as most will included as separate files

\section*{Figures}
  \subsection*{Figure 1 - Sample figure title}
      A short description of the figure content
      should go here.

  \subsection*{Figure 2 - Sample figure title}
      Figure legend text.



%%%%%%%%%%%%%%%%%%%%%%%%%%%%%%%%%%%
%%                               %%
%% Tables                        %%
%%                               %%
%%%%%%%%%%%%%%%%%%%%%%%%%%%%%%%%%%%

%% Use of \listoftables is discouraged.
%%
\section*{Tables}
  \subsection*{Table 1 - Sample table title}
    Here is an example of a \emph{small} table in \LaTeX\ using  
    \verb|\tabular{...}|. This is where the description of the table 
    should go. \par \mbox{}
    \par
    \mbox{
      \begin{tabular}{|c|c|c|}
        \hline \multicolumn{3}{|c|}{My Table}\\ \hline
        A1 & B2  & C3 \\ \hline
        A2 & ... & .. \\ \hline
        A3 & ..  & .  \\ \hline
      \end{tabular}
      }
  \subsection*{Table 2 - Sample table title}
    Large tables are attached as separate files but should
    still be described here.



%%%%%%%%%%%%%%%%%%%%%%%%%%%%%%%%%%%
%%                               %%
%% Additional Files              %%
%%                               %%
%%%%%%%%%%%%%%%%%%%%%%%%%%%%%%%%%%%

\section*{Additional Files}
  \subsection*{Additional file 1 --- Sample additional file title}
    Additional file descriptions text (including details of how to
    view the file, if it is in a non-standard format or the file extension).  This might
    refer to a multi-page table or a figure.

  \subsection*{Additional file 2 --- Sample additional file title}
    Additional file descriptions text.


\end{bmcformat}
\end{document}







=======
%% BioMed_Central_Tex_Template_v1.05
%%                                      %
%  bmc_article.tex            ver: 1.05 %
%                                       %


%%%%%%%%%%%%%%%%%%%%%%%%%%%%%%%%%%%%%%%%%
%%                                     %%
%%  LaTeX template for BioMed Central  %%
%%     journal article submissions     %%
%%                                     %%
%%         <27 January 2006>           %%
%%                                     %%
%%                                     %%
%% Uses:                               %%
%% cite.sty, url.sty, bmc_article.cls  %%
%% ifthen.sty. multicol.sty		       %%
%%									   %%
%%                                     %%
%%%%%%%%%%%%%%%%%%%%%%%%%%%%%%%%%%%%%%%%%


%%%%%%%%%%%%%%%%%%%%%%%%%%%%%%%%%%%%%%%%%%%%%%%%%%%%%%%%%%%%%%%%%%%%%
%%                                                                 %%	
%% For instructions on how to fill out this Tex template           %%
%% document please refer to Readme.pdf and the instructions for    %%
%% authors page on the biomed central website                      %%
%% http://www.biomedcentral.com/info/authors/                      %%
%%                                                                 %%
%% Please do not use \input{...} to include other tex files.       %%
%% Submit your LaTeX manuscript as one .tex document.              %%
%%                                                                 %%
%% All additional figures and files should be attached             %%
%% separately and not embedded in the \TeX\ document itself.       %%
%%                                                                 %%
%% BioMed Central currently use the MikTex distribution of         %%
%% TeX for Windows) of TeX and LaTeX.  This is available from      %%
%% http://www.miktex.org                                           %%
%%                                                                 %%
%%%%%%%%%%%%%%%%%%%%%%%%%%%%%%%%%%%%%%%%%%%%%%%%%%%%%%%%%%%%%%%%%%%%%


\NeedsTeXFormat{LaTeX2e}[1995/12/01]
\documentclass[10pt]{bmc_article}    



% Load packages
\usepackage{cite} % Make references as [1-4], not [1,2,3,4]
\usepackage{url}  % Formatting web addresses  
\usepackage{ifthen}  % Conditional 
\usepackage{multicol}   %Columns
\usepackage[utf8]{inputenc} %unicode support
%\usepackage[applemac]{inputenc} %applemac support if unicode package fails
%\usepackage[latin1]{inputenc} %UNIX support if unicode package fails
\urlstyle{rm}
 
 
%%%%%%%%%%%%%%%%%%%%%%%%%%%%%%%%%%%%%%%%%%%%%%%%%	
%%                                             %%
%%  If you wish to display your graphics for   %%
%%  your own use using includegraphic or       %%
%%  includegraphics, then comment out the      %%
%%  following two lines of code.               %%   
%%  NB: These line *must* be included when     %%
%%  submitting to BMC.                         %% 
%%  All figure files must be submitted as      %%
%%  separate graphics through the BMC          %%
%%  submission process, not included in the    %% 
%%  submitted article.                         %% 
%%                                             %%
%%%%%%%%%%%%%%%%%%%%%%%%%%%%%%%%%%%%%%%%%%%%%%%%%                     


\def\includegraphic{}
\def\includegraphics{}



\setlength{\topmargin}{0.0cm}
\setlength{\textheight}{21.5cm}
\setlength{\oddsidemargin}{0cm} 
\setlength{\textwidth}{16.5cm}
\setlength{\columnsep}{0.6cm}

\newboolean{publ}

%%%%%%%%%%%%%%%%%%%%%%%%%%%%%%%%%%%%%%%%%%%%%%%%%%
%%                                              %%
%% You may change the following style settings  %%
%% Should you wish to format your article       %%
%% in a publication style for printing out and  %%
%% sharing with colleagues, but ensure that     %%
%% before submitting to BMC that the style is   %%
%% returned to the Review style setting.        %%
%%                                              %%
%%%%%%%%%%%%%%%%%%%%%%%%%%%%%%%%%%%%%%%%%%%%%%%%%%
 

%Review style settings
\newenvironment{bmcformat}{\begin{raggedright}\baselineskip20pt\sloppy\setboolean{publ}{false}}{\end{raggedright}\baselineskip20pt\sloppy}

%Publication style settings
%\newenvironment{bmcformat}{\fussy\setboolean{publ}{true}}{\fussy}



% Begin ...
\begin{document}
\begin{bmcformat}


%%%%%%%%%%%%%%%%%%%%%%%%%%%%%%%%%%%%%%%%%%%%%%
%%                                          %%
%% Enter the title of your article here     %%
%%                                          %%
%%%%%%%%%%%%%%%%%%%%%%%%%%%%%%%%%%%%%%%%%%%%%%

\title{Source of Functional Annotations in UniProtKB and implications for
Function Prediction}
 
%%%%%%%%%%%%%%%%%%%%%%%%%%%%%%%%%%%%%%%%%%%%%%
%%                                          %%
%% Enter the authors here                   %%
%%                                          %%
%% Ensure \and is entered between all but   %%
%% the last two authors. This will be       %%
%% replaced by a comma in the final article %%
%%                                          %%
%% Ensure there are no trailing spaces at   %% 
%% the ends of the lines                    %%     	
%%                                          %%
%%%%%%%%%%%%%%%%%%%%%%%%%%%%%%%%%%%%%%%%%%%%%%


\author{Alexandra Schnoes$^1$%
       \email{Alexandra Schnoes - schnoes@ucsf.edu}%
      \and
         Alexander Thorman$^2$%
         \email{Alexander Thorman - thormanaaw@muohio.edu}
       \and 
         Patricia C Babbitt$^1$%
         \email{Patricia C Babbitt - babbitt@ucsf.edu}%
        and
         Iddo Friedberg\correspondingauthor$^{2,3}$%
        \email{Iddo Friedberg\correspondingauthor - i.friedberg@muohio.edu}
      }
      

%%%%%%%%%%%%%%%%%%%%%%%%%%%%%%%%%%%%%%%%%%%%%%
%%                                          %%
%% Enter the authors' addresses here        %%
%%                                          %%
%%%%%%%%%%%%%%%%%%%%%%%%%%%%%%%%%%%%%%%%%%%%%%

\address{%
    \iid(1)UCSF,%
        San Francisco, CA, USA\\
    \iid(2)Department of Microbiology, Miami University, Oxford, OH USA\\
    \iid(3)Department of Computer Science and Software Engineering , Miami University, Oxford, OH USA
}%

\maketitle

%%%%%%%%%%%%%%%%%%%%%%%%%%%%%%%%%%%%%%%%%%%%%%
%%                                          %%
%% The Abstract begins here                 %%
%%                                          %%
%% The Section headings here are those for  %%
%% a Research article submitted to a        %%
%% BMC-Series journal.                      %%  
%%                                          %%
%% If your article is not of this type,     %%
%% then refer to the Instructions for       %%
%% authors on http://www.biomedcentral.com  %%
%% and change the section headings          %%
%% accordingly.                             %%   
%%                                          %%
%%%%%%%%%%%%%%%%%%%%%%%%%%%%%%%%%%%%%%%%%%%%%%


\begin{abstract}
        % Do not use inserted blank lines (ie \\) until main body of text.
        \paragraph*{Background:} Computational protein function
prediction programs rely upon well-annotated databases for training
their algorithms. These databases, in turn, rely upon the work of curators
applying experimental findings from scientific literature to protein
sequence data. However, due to various high-throughput experimental
assays, there are a few papers that dominate the protein annotations.
Here we investigate just how prevalent is the ``few papers --
many proteins'' bias. We discuss how this bias affects our view of the
protein function universe, and consequently our ability to predict
protein function.
      
        \paragraph*{Results:} We examine the annotation of UniProtKB by the Gene
Ontology Annotation project (GOA), and show that the distribution of proteins
annotated per paper follows a scale-free distribution, with X papers dominating X\%
of the annotations. Since each of the dominant papers describes the use of an assay
that can find only one function or a small group of functions, this leads to a
substantial bias in what we know about the function of many proteins.

        \paragraph*{Conclusions:} Given the experimental techniques
available, the protein function annotation bias is unavoidable. Knowing
that this bias exists and understanding its extent is important for
database curators, developers of function annotation programs, and
anyone who uses protein function annotation data to plan experiments.

\end{abstract}



\ifthenelse{\boolean{publ}}{\begin{multicols}{2}}{}




%%%%%%%%%%%%%%%%%%%%%%%%%%%%%%%%%%%%%%%%%%%%%%
%%                                          %%
%% The Main Body begins here                %%
%%                                          %%
%% The Section headings here are those for  %%
%% a Research article submitted to a        %%
%% BMC-Series journal.                      %%  
%%                                          %%
%% If your article is not of this type,     %%
%% then refer to the instructions for       %%
%% authors on:                              %%
%% http://www.biomedcentral.com/info/authors%%
%% and change the section headings          %%
%% accordingly.                             %% 
%%                                          %%
%% See the Results and Discussion section   %%
%% for details on how to create sub-sections%%
%%                                          %%
%% use \cite{...} to cite references        %%
%%  \cite{koon} and                         %%
%%  \cite{oreg,khar,zvai,xjon,schn,pond}    %%
%%  \nocite{smith,marg,hunn,advi,koha,mouse}%%
%%                                          %%
%%%%%%%%%%%%%%%%%%%%%%%%%%%%%%%%%%%%%%%%%%%%%%




%%%%%%%%%%%%%%%%
%% Background %%
%%
\section*{Background}
 Text for this section.

Functional annotation of proteins is a cardinal challenge in molecular biology today. It has
been said that while we may soon reach the \$1,000 genome (and for some prokaryotes, indeed
we have), we will still need to pay \$20,000 analysis bill. There are and estimated XXX
non-redundant open reading frame sequences, but less than 40\% are functionally annotated.
Moreover, the majority of functional annotations 


 
%%%%%%%%%%%%%%%%%%%%%%%%%%%%
%% Results and Discussion %%
%%
\section*{Results and Discussion}
  \subsection*{Papers and proteins}

    As described in the Background section, with the advent of
high-throughput experiments it has become possible to conduct
large-scale interrogations of protein functions. Some papers therefore
reveal one or more functional aspects of a large amount of proteins
which respond to the particular type of interrogation conducted. To
understand how prevalent this phenomenon is, we looked at the UniprotKB
gene ontology annotation files, or UniProt GOA. UniProtKB is annotated
by GO terms both manually and automatically using an exacting procedure
described in \cite{UniprotKB-GOA}. Briefly, there is a six-step
procedure which includes sequence curation, sequence motif analyses,
literature-based curation, reciprocal BLAST\cite{BLAST} searches,
attribution of all resources leading to the included findings, and a
quality assurance phase. If the annotation source is a research article, the
attribution includes a PubMed ID.  For each GO term associated with a
protein, there is also an \textit{evidence code} with which is used to
explain how the association between the protein and the GO term was
made. Experimental evidence codes include such terms as: Inferred by
Direct Assay (IDA) which indicates that ``a direct assay was carried out
to determine the function, process, or component indicated by the GO
term'' or \textit{Inferred from Physical Interaction} (IPI) which ``Covers
physical interactions between the gene product of interest and another
molecule.'' (Quotes are from the GO site, geneontology.org). The
computational analysis evidence codes are generally considered less
reliable than the experimental ones, and include terms such as
\textit{Inferred from Sequence or Structural Similarity} (ISS) and
\textit{Inferred from Sequence Orthology} (ISO). However, these are still
assigned by a curator. There are also
non-computational and non-experimental evidence codes, the most
prevalent being \textit{Inferred from Electronic Annotation} (IEA) ``Used
for annotations that depend directly on computation or automated
transfer of annotations from a database''. IEA evidence means that the
annotation was not made by a curator, and is not checked manually. 

Different degrees of reliability are associated with the  evidence codes, with
experimental codes considered to be of higher reliability than non-experimental
codes. However, due to an ongoing increase in high-throughput experimental
methods, we suspected that high-throughput experiments may dominate the protein
annotation landscape. 

To test our hypothesis, we first examined assignments with two degrees of reliability:
experimental evidence codes (EXP, IDA, IPI, IMP, IGI, IEP), and all others. The
reason for this partition into two groups is that we wanted to know which, if
any, high throughput experiments dominate the protein function annotation
landscape. This we compared with a baseline of annotations of all types of
evidence, and with those which are non-experimental. The results are shown in
Figure\ref{fig:papers-prots}.

As can be seen in Figure\ref{fig:papers-prots}, the distribution of proteins
annotated per paper follows a scale-free distribution.  Many proteins are annotated by a
few papers. (ZZZNeed to look at the percentage of proteins annotated by the top
10, top 20 papers.) We therefore conclude that there is indeed a bias in
experimental annotations, in which there are few papers that annotate a large
number of proteins.

Our next question was how much annotation bias these papers introduce. We decided
to look who are the top contributing papers to protein annotation in
UniProtKB-GOA, and what type of GO terms they contribute.  The results to the
first question are summarized in Table\ref{table:top-papers}.  As can be seen,
each paper is specific to a single species (typically a model
organism) and assay that is used to annotate the proteins in that organism. Since
a single assay was used, then typically only one ontology (MF, BP or CC) was
annotated. 
 
\subsection*{Annotation bias in model organisms}

To see how much a single species-- and method-- specific large-scale assay affects
the entire annotation of a species, we examined the relative contribution of each
paper to the entire corpus of experimentally annotated protein in that species.  All
the species we examined were model organisms, as all the top annotation-contributing
papers dealt with model organisms.  The results are summarized in
Figure\ref{fig:rel-contrib}.

\subsection*{Annotation quality}

One reflection on annotation coverage is the number of GO terms assigned to any
given protein. Ostensibly, the larger the number of GO terms that are assigned to a
protein, the more comprehensive its annotation, provided that these terms are
non-redundant, i.e. not direct parents of each other. 

The median number of annotations per protein was 1.09, which means that most of the top-50
papers do not provide more than a single GO-term per annotation. Also, the mean number of
annotations per protein was 1.59. The difference between the mean and median reflects that
most papers have an annotations-per-protein ratio which is close to 1, whereas few have a
higher annotation ratio. As shown in Figure~\ref{fig:annotation-ratio}, that is essentially
the case. Seven papers had an annotations per proteins ratio which is above 2. Those
were paper numbers 1, 4, 10, 23, 34 and 43 on the list. We decided to examine these papers
to see whether these studies did indeed provide better coverage, and what distinguished them
from the other high throughput studies. To follow are brief summaries of the
methodologies used in these papers.

\textbf{Toward a confocal subcellular atlas of the human proteome.}

In this microscopy-based study, the authors used specific staining techniques and
confocal microscopy to describe the subcellular localization of 4937 proteins, which
were each assigned to one or more of ten different subcellular compartments. Since
proteins may be assigned to more than a single compartment, there is a mean of 2.23
GO annotations per protein. 
\cite{PMID:11121744}. 


It appears that most of the high-annotating papers are 



%%%%%%%%%%%%%%%%%%%%%%
\section*{Conclusions}
  Text for this section \ldots


  
%%%%%%%%%%%%%%%%%%
\section*{Methods}
  \subsection*{Databases used}
    Text for this sub-section \ldots

  \subsection*{Another methods sub-heading for this section}
    Text for this sub-section \ldots

  \subsection*{Yet another sub-heading for this section}
    Text for this sub-section \ldots


    
%%%%%%%%%%%%%%%%%%%%%%%%%%%%%%%%
\section*{Authors contributions}
    Text for this section \ldots

    

%%%%%%%%%%%%%%%%%%%%%%%%%%%
\section*{Acknowledgements}
  \ifthenelse{\boolean{publ}}{\small}{}
  Text for this section \ldots


 
%%%%%%%%%%%%%%%%%%%%%%%%%%%%%%%%%%%%%%%%%%%%%%%%%%%%%%%%%%%%%
%%                  The Bibliography                       %%
%%                                                         %%              
%%  Bmc_article.bst  will be used to                       %%
%%  create a .BBL file for submission, which includes      %%
%%  XML structured for BMC.                                %%
%%                                                         %%
%%                                                         %%
%%  Note that the displayed Bibliography will not          %% 
%%  necessarily be rendered by Latex exactly as specified  %%
%%  in the online Instructions for Authors.                %% 
%%                                                         %%
%%%%%%%%%%%%%%%%%%%%%%%%%%%%%%%%%%%%%%%%%%%%%%%%%%%%%%%%%%%%%


{\ifthenelse{\boolean{publ}}{\footnotesize}{\small}
 \bibliographystyle{bmc_article}  % Style BST file
  \bibliography{bmc_article} }     % Bibliography file (usually '*.bib' ) 

%%%%%%%%%%%

\ifthenelse{\boolean{publ}}{\end{multicols}}{}

%%%%%%%%%%%%%%%%%%%%%%%%%%%%%%%%%%%
%%                               %%
%% Figures                       %%
%%                               %%
%% NB: this is for captions and  %%
%% Titles. All graphics must be  %%
%% submitted separately and NOT  %%
%% included in the Tex document  %%
%%                               %%
%%%%%%%%%%%%%%%%%%%%%%%%%%%%%%%%%%%

%%
%% Do not use \listoffigures as most will included as separate files

\section*{Figures}
  \subsection*{Figure 1 - Sample figure title}
      A short description of the figure content
      should go here.

  \subsection*{Figure 2 - Sample figure title}
      Figure legend text.



%%%%%%%%%%%%%%%%%%%%%%%%%%%%%%%%%%%
%%                               %%
%% Tables                        %%
%%                               %%
%%%%%%%%%%%%%%%%%%%%%%%%%%%%%%%%%%%

%% Use of \listoftables is discouraged.
%%
\section*{Tables}
  \subsection*{Table 1 - Sample table title}
    Here is an example of a \emph{small} table in \LaTeX\ using  
    \verb|\tabular{...}|. This is where the description of the table 
    should go. \par \mbox{}
    \par
    \mbox{
      \begin{tabular}{|c|c|c|}
        \hline \multicolumn{3}{|c|}{My Table}\\ \hline
        A1 & B2  & C3 \\ \hline
        A2 & ... & .. \\ \hline
        A3 & ..  & .  \\ \hline
      \end{tabular}
      }
  \subsection*{Table 2 - Sample table title}
    Large tables are attached as separate files but should
    still be described here.



%%%%%%%%%%%%%%%%%%%%%%%%%%%%%%%%%%%
%%                               %%
%% Additional Files              %%
%%                               %%
%%%%%%%%%%%%%%%%%%%%%%%%%%%%%%%%%%%

\section*{Additional Files}
  \subsection*{Additional file 1 --- Sample additional file title}
    Additional file descriptions text (including details of how to
    view the file, if it is in a non-standard format or the file extension).  This might
    refer to a multi-page table or a figure.

  \subsection*{Additional file 2 --- Sample additional file title}
    Additional file descriptions text.


\end{bmcformat}
\end{document}







>>>>>>> upstream/master
